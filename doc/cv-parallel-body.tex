\documentclass{beamer}

%\beamertemplatesolidbackgroundcolor{black!5}
%\beamertemplatetransparentcovered

\usepackage{amsmath}
\usepackage{amssymb}

\title{Parallelisierung in Computer Vision}
\subtitle{Grundlagen mit Implementierungsbeispielen}

\author[Braun]{Christian Braun}

\begin{document}

\begin{frame}
  \titlepage
\end{frame}

\begin{frame}
  \frametitle{Übersicht}
  \tableofcontents
\end{frame}

\begin{frame}
  \frametitle{Konventationen bezüglich Computer Vision}
  \framesubtitle{Um eine gemeinsame Grundlage zu schaffen}
  \begin{itemize}
    \item Abbildungsfunktionen können ein n-Tupel auf ein m-Tupel abbilden, mit n,m $\in \mathbb{N}$
    \item Ein Tupel besteht aus einer Matrix(c,r) von Pixeln, für die n = c*r gilt mit c,r $\in \mathbb{N} \&$ c,r $\geq$ 1
    \item Ein Pixel ist ein 3-wertiges Skalar welches die Farben in Reihenfolge Blau, Grün und Rot darstellt
    \item Ein (Kamera-)Bild ist eine Matrix(c,r) 
  \end{itemize}
\end{frame}

\section{Einführung}
\subsection{Beispiele aus der Computer Vision}
\begin{frame}<1>
  \frametitle{Beispiele aus der Computer Vision}
  %\framesubtitle{}
  \begin{itemize}
    \item Korrektur von Abbildungsfehlern
    \item Skalierung eines Bildes
  \end{itemize}
\end{frame}

\begin{frame}<2>
  \frametitle{Unterschiedliche Arten von Parallelisierung}
  \framesubtitle{Bei der Parallelisierung sind zwei Begebenheiten zu unterscheiden}
  \begin{itemize}
    \item Ob eine einzelne Aufgabe auf mehrere aufgeteilt werden soll, 
    \item oder mehrere von einander unabhängige Aufgaben parallelisiert werden sollen.
  \end{itemize}

\end{frame}

\begin{frame}<3>
  \frametitle{Selbststehende Aufgabe aufteilen}
  \framesubtitle{Die zu untersuchenden Eigenschaften}

  \begin{itemize}
    \item U ist die Menge der Ursprungspositionen im Speicher
    \item Z ist die Menge der Zielpositionen im Speicher
    \item $\digamma$ ist die Funktion, die U auf Z abbildet
  \end{itemize}

  \begin{itemize}
    \item Die Größe und Position eines jeden einzelnen Elementes der Eingabemenge ist bekannt
    \item Die Größe und Position eines jeden einzelnen Elementes der Ausgabemenge ist bekannt
    \item Die Bearbeitungszeit eines Elementes
  \end{itemize}

\end{frame}

\begin{frame}<4>
  \frametitle{Beispiel: Skalierung eines Bildes}
  \framesubtitle{Halbierung des Bildes wird verwendet um in einer geringeren Datenmenge Bereiche auszuschließen}

  \begin{itemize}
    \item Pixel werden in einem n-Tupel gebündelt, wobei n eine endliche Zahl ist 
    \item Beispiel: Wird ein Bild halbiert, wird ein 4-Tupel auf ein 1-Tupel abgebildet. Dabei werden 2x2 Pixel zu einem zusammengefasst
  \end{itemize}

  \begin{itemize}
    \item Ist die Positionsänderung eines Pixels injektiv, wird jeder Pixel aus U genau einem Pixel aus Z zugewiesen.

    \item ist die Positionsänderung eines Pixel dagegen surjektiv, kann ein Pixel aus Z durch mehrere Pixel aus U beschrieben werden.
  \end{itemize}

\end{frame}

\end{document}
